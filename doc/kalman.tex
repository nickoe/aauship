\section{Position estimation}
To give a better position estimate which can be fed to the controller as well, the different data collected from the sensors mounted are put through a Kalman filter. This filter takes the different measurements as inputs, and uses these to give a better estimate of the position, rather than the quite noisy measurements taken using just the raw \ac{GPS} data. 

To develop such a filter, the model of the ship is to be computed, as well as a mapping of the different inputs and outputs to the system. The model of the forward and sidewards case (surge and sway) are the same as for the discrete system. 

\subsection{State model}
The state model is used as a base for computing the influence the different inputs have on the system. The matrix is the same as the $\vec{A}$ matrix used to describe the state space representation of the system. The general state expression is given as:
\begin{align}
\vec{Y}(n) = \vec{A}(n)\vec{Y}(n-1) + \vec{Z}(n)
\end{align}
\noindent Where:
\begin{ffk}
$\vec{A}(n)$ is the state matrix\\
$\vec{Y}(n-1)$ is the last input to the system\\
$\vec{Z}(n)$ is the driving noise
\end{ffk}
In this case, the driving noise $\vec{Z}(n)$ will be the inputs to the system, which can then be used to estimate the different states. The states to be estimated is the velocity $\dot{x}$, the angular velocity $\omega$ and the angle of the vessel in the local frame $\theta$. The driving noise (or input) can be defined as the $\vec{B}$ matrix in the state system, multiplied with the different inputs given to the system, namely $n_1$ and $n_2$. When inserted, the formula for the state model becomes:
\begin{align}
\vec{Y}(n) = \vec{A}(n)\vec{Y}(n-1) + \vec{B}\vec{u}
\end{align}
And when terms are inserted, the formula becomes: 
\begin{align}
\vec{Y}(n) = \vec{A}(n)\vec{Y}(n-1) + \vec{B}\vec{u}
\end{align}\todo{insert true formula -lunde}

\subsection{Observation model}
The observation model, is a model that models the different observations. In this case, the different observations are measured directly, as we can measure both the angular velocity, the angle and the velocity of the craft. The general formula for the observation model is given as:
\begin{align}
\vec{X}(n) = \vec{H}(n)\vec{Y}(n) + \vec{W}(n)
\end{align}
\noindent Where:
\begin{ffk}
$\vec{H}(n)$ is the model linking the measurements to the observations\\
$\vec{W}(n)$ is the noise from the measurements
\end{ffk}
The noise from the measurements is estimated using previous measurements which can be used to estimate the variance and the mean of the measurements. The noise can in general be seen as zero-mean Gaussian white noise processes, which makes for the assumption:
\begin{align}
\vec{W}(n) \sim \mathcal{N}(0,\sigma_Z^2)
\end{align}
As $\vec{Y}(n)$ is a row vector, $\vec{W}(n)$ is also a row vector with the same dimension. This calls for different variances on the different noise additions, for each of the measurements. As the variance of the noise on the \ac{IMU} is a lot bigger than on the \ac{GPS}. As all the measurements are available directly, the $\vec{H}(n)$ matrix is equal to identity. Giving the final observation model:
\begin{align}
\vec{X}(n) = \vec{Y}(n) + \vec{W}(n)
\end{align}

% Wouldn't it be a fair assumption that a GPS doesn't have zero-mean, but has a wandering mean that would wander over time? 

% Text about the vector Kalman filer
% Text about the covariance matrix of such a system
\subsection{The Covariance matrix}
As the Kalman filter is given as a vector Kalman filter, the covariance matrix is to be computed. The definition for a covariance matrix is given as:
\begin{align}
\mathcal{C} = \text{E}\langle[x(t_1) - \text{E}[x(t_1)]][x(t_2) - \text{E}[x(t_2)]]^\text{T}\rangle
\end{align}
% Text about some simulations of the system
% Text about linearizations / further computation.

\subsection{Position estimation}
This section will contain the different things we've considered during the miniproject in Kalman filtering, and will be used to give a better estimate of the actual position!
Inputs from the \ac{GPS} (position x,y) and velocity
Inputs from the \ac{IMU}
Basically a description of the miniproject. 