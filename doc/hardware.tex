\chapter{Hardware considerations}

Because of the practical nature of the project, a wide range of components had to be chosen. There are several things to be taken into consideration when designing a scale ship from scratch

\section{LLI electronics}
The \ac{LLI} electronics is a plug and play device that controls several smaller ship instances. It provides outputs to control the actuators and motors as well as basic sensor readings to the \ac{HLI}.

The \ac{LLI} has been designed to allow for the above functionality. As described in section~\vref{sec:platform} the \ac{LLI} is an embedded platform, for which an Atmel AVR microcontroller has been chosen. The main control board is an Arduino Mega 2560, and a shield interface to all peripherals has been designed~\vref{chap:schema}.

The final LLI supports the following features:
\begin{description}
\item[Serial]\hfill \\ interface to the \ac{HLI} with a baud rate of 115200 bps
\item[PWM]\hfill \\ outputs for actuators
\item[I$^2$C]\hfill \\ option for aux communication
\item[Analog]\hfill \\ inputs for various sensors
\item[Relay driver]\hfill \\ output
\item[5V]\hfill \\ regulated output
\end{description}

More detailed information about hardware layout~\vref{fig:lli-hw} and the hardware schematic~\vref{chap:schema} can be found in the appendix.

\section{Lithium Polymer batteries}

	Lithium Polymer batteries have been chosen for powering the boat motors as well as for all the electronics, because they offer a very high specific energy and energy density, and can be charged in a relatively short period of time. 

\subsection{Energy calculations}
	
	Each battery is composed of 4 series connected cells with a nominal voltage of 3.7 V, making a total of 14.8 V per battery. There will be 6 such units, each storing 3200 mAh worth of charge, which yields a total of $ 6 \cdot 14.8 \text{ V} \cdot 3.2\text{ Ah} = 284.16 \text{ Wh} $, around 1 MJ of energy.
	
	There are two separate electrical circuits: a power circuit for the motors for which there are 4 dedicated batteries connected in parallel and 2 in the electronics circuit which is separated in order to avoid noise in the digital circuitry. These systems are alloted 198.44 Wh and 94.72 Wh respectively.
	
	If the batteries run at full power (draining a maximum of 80 Amps) the ship will be able to sail for 5 minutes, however, a scenario where the engine runs at full power with a brake on the propellers is extremely unlikely.
	
	Measurements show that the engines should be ran at a setting of 60 of the maximum range of 500 (discretized control steps) and that in this case they drain 2.5 A combined. This will give an operation time of 5.1 hours - thus giving quite a big range on the ships. Assuming a 1 m/s velocity kept constant for 2.5 hours, this gives a range of 9 kilometers forth and 9 kilometers back or being able to measure an area of 1 x 1 km in strips 100 meters apart and still have surplus power.
	
	\subsection{Battery care and charge meters}
	
	Due to the delicate nature of Li-Po batteries, it is absolutely required to never over-charge or over-discharge them because there is a high risk of permanent damage to the batteries. If the temperature continues to raise, there is even a risk of fire and/or explosion. In order to prevent this, we are using a dedicated Li-Po charger with an included balancer, which ensures that none of the cells in the batteries go above the absolute maximum of 4.2 V while charging.
	
	The problem of over-discharge is solved in the power circuit due to the fact that the brushless motor controller has a safety switch which does not allow any cell to drop below 3 V, which would cause permanent damage to the batteries, as they cannot recover after being over-discharged. 
	
	The electronic circuit can drain the batteries more than the maximum limit though. In order to prevent this from happening, we implemented a couple of battery level monitors that are integrated in the \ac{LLI} module and, which can compute the battery levels, thus allow the user to retrieve the boat in order to charge it. They should also include a master kill switch that can save the batteries as well.

\section{Electronics design}

	\subsection{Bow thruster controller}
	\label{subsec:bow thruster controller}
	
	The bow thruster controller is a circuit whose function is to provide power to the bow thruster, under the control of the \ac{LLI}. This uses a L298 H-bridge connected to a logic gate, so that it can be driven with just two signals: direction and \ac{PWM}.
	
	Another use of this circuit is to provide a lossless voltage level conversion from the batteries' nominal 14.8V to the motor's 7.2V rated voltage. That means that the maximum theoretical duty cycle of the PWM bow thruster control signal is $ 14.8 / 7.2 = 48 \% $. This, however, is not the actual value that will be used, due to the fall time of the L298 chip, which is pretty big at the 1kHz PWM frequency. By using an oscilloscope to measure the True RMS value of the electronic circuit's output, it was empirically determined that the maximum pulse width value is 33\%. 
	The motor used to drive the bow thruster also has a minimum starting voltage for which the PWM minimum duty cycle was empirically determined to be around 5\% in this case.
	
	Since the L298 chip has two H bridges inside it, the designed board which can be viewed in the appendix \vref{appendices:bow thruster schematic} includes the circuitry and parts for the two controllers, since it would provide redundancy in case one of them is overloaded or otherwise stops functioning.
	
\section{Motors and propellers}

As the ship should be able to cope with rather powerful streams a powerful propulsion system had to be chosen. The main propulsion unit consists of 2 Grapuner Brushless 750 14.8 V 1200W engines - which together produce around 3 HP at maximum input. With these powerful engines we have a large dynamic range, that allows for fast acceleration and a fast maneuvering, as well as navigating in strong currents. 

The propellers are 2 counterrotating Raboesch brass propellers with a diameter of 60 mm. 