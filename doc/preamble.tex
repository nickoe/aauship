\documentclass[a4paper,11pt,twoside,fleqn,openright]{memoir}
 
% Pakker
\usepackage[utf8]{inputenc} % Så må vi bruge æ, ø og å
%\usepackage[ansinew]{inputenc}
%\usepackage[danish]{babel} % Dansk opsætning
\usepackage[T1]{fontenc} % Hjælper med ordeling ved æ, ø og å. Sætter fontene til at være ps-fonte i stedet for bmp.
%\usepackage[scaled]{beramono} % Bedre monospace font
\usepackage{amsmath,amsfonts,amssymb} % God matematik
\usepackage{sistyle} % Enheder til fysik
\usepackage{array,booktabs} % Til gode tabeller
\usepackage{ragged2e} % For at kunnu lave tabeller med fast kolonnebredde, bruges sammen med 'array'
\usepackage{float} % Vi må nu bruge H som placering til floats
\usepackage{hhline} % Sum linie i tabel
\usepackage{multirow} % Fletning af rækker
\usepackage{multicol} % Fletning af kolonner
\usepackage{xcolor} % Vi kan bruge \color osv
\usepackage{colortbl} % Muligøre farver i tabeller
%\usepackage[danish=quotes]{csquotes} % Danske anførselstegn, brug enquote{}
\usepackage[english=british]{csquotes} 
\usepackage{graphicx} % Inkludere ekstern grafik
\usepackage[english,final]{varioref} % Vi kan anvende \vref
\usepackage{natbib} % Bedre litteratur henvisninger
\usepackage{pdfpages} % Inkludere en pdf side som en side  
\usepackage{acronym} % Smart akronymhåndtering
\usepackage{suffix} % Krævet af acronym
\usepackage{listings} % Til at inkludere kildekode direkte
\usepackage{lipsum} % Lorem ipsum dolar sit amet
\usepackage{caption} % Vi kan bruge \captionof
\usepackage{subfig} % Vi kan nu bruge \subfloat
\usepackage{calc} % Vi kan regne med tællere
\usepackage{changepage} % Vi kan ændre sidelayoutet lokalt
\usepackage{layout} % Vi kan se dimensionecne på vores layout med \layout
\usepackage[footnote,final,english,silent,nomargin]{fixme}
\usepackage[colorinlistoftodos]{todonotes}
\usepackage[pdftex,bookmarks=true,bookmarksnumbered=true]{hyperref} % Links i dokumentet
\usepackage{textcomp} % HVAD FANDEN GØR DENNE PAKKE?
\usepackage{threeparttable} % Så vi kan lave tablenotes (se latexbogen)
\usepackage{fixltx2e} % HVAD FANDEN GØR DENNE PAKKE?
\usepackage{minitoc} % Vi kan lave del inholdsfortegnelser forhåbentlig
\usepackage{color} % For use in the hilighting funcion

% For sjov pakker
\usepackage{marvosym}
\usepackage{wasysym}
%\usepackage{fourier}

\captionsetup{font={small,sf},labelfont=bf}

\pagestyle{companion}

% Litteraturliste stil
\bibliographystyle{plainnat}
%\bibliographystyle{abbrvnat}
%\bibliographystyle{unsrtnat}

% Andre egensakber for komandoer
\newcommand\Cpp{C\raisebox{\height / 2}[\height][\depth]{\tiny ++}} % Pænere C++
\newcommand{\mimg}[4]{\marginpar{\small \centering{\includegraphics[width=#1]{#2}\captionof{figure}{\newline #3}\label{#4}}}} % Margin billede
\newcommand{\lmimg}[4]{\marginpar{\small\centering{% Latex margin billede
  \def\svgwidth{#1}
  \graphicspath{{illustrations/}}
  \input{illustrations/#2.pdf_tex}
  \captionof{figure}{\newline #3}
  \label{#4}}}
}
\newcommand{\mnote}[1]{\marginpar{\small \textsf{\textbf{Note}\\{#1}}}} % Margin note
\newcommand{\mcd}[1]{\marginpar{\textsf{\textbf{\includegraphics[height=11pt]{formalities/media-optical}}\\\url{#1}}}} % CD-rom henvisning i margin
\newcommand{\cd}[1]{\includegraphics[height=9pt]{formalities/media-optical}\url{#1}} % CD-rom henvisning i tekst
\newcommand{\limg}[5]{\begin{figure}[#1]
  \centering
  \def\svgwidth{#2}
  \graphicspath{{illustrations/}}
  \input{illustrations/#3.pdf_tex}
  \caption{#4}
	\label{#5}
\end{figure}}
\newcommand{\head}[1]{{\slshape{#1}}\vspace{5mm}} % Header
\newcommand{\tail}{\vspace{3mm}\fancybreak{$*\quad*\quad*$}\vspace{3mm}} % Røvhul
\newcommand{\enhed}[1]{\hfill\hbox{[#1]}\qquad} % At indsatte heb enheder i firkantparentes i "equation"
\newcommand{\dB}[0]{\hbox{dB}} % At skrive dB oprejst
%\addto\captionsdanish{
%\renewcommand\appendixname{Appendiks}
%\renewcommand\contentsname{Indholdsfortegnelse}

% Make vectors bold instead of a arrow
\let\oldhat\hat
\renewcommand{\vec}[1]{\mathbf{#1}}
\renewcommand{\hat}[1]{\oldhat{\mathbf{#1}}}

\makechapterstyle{box}{
  \renewcommand*{\printchaptername}{}
  \renewcommand*{\chapnumfont}{\normalfont\sffamily\huge\bfseries}
  \renewcommand*{\printchapternum}{
    \flushleft
    \begin{tikzpicture}[line width=4pt]
      \draw (0,1) -- (0,0) -- (1,0);
      \draw (2,1) -- (2,2) -- (1,2);
      \draw[color=gray] (1cm,1cm) node { \chapnumfont\thechapter };
    \end{tikzpicture}
  }
  \renewcommand*{\chaptitlefont}{\normalfont\sffamily\Huge\bfseries}
  \renewcommand*{\printchaptertitle}[1]{\flushleft\chaptitlefont##1}
  \setlength\beforechapskip{-100pt}
}

\newif\ifchapternonum
\makechapterstyle{nickoe}{
\renewcommand\printchapternonum{\chapternonumtrue}
  \renewcommand*{\printchaptername}{} % Removes the 'Chapter' text
  \renewcommand*{\chapnumfont}{\fontfamily{pbk}\fontseries{m} \fontshape{n}\fontsize{80}{35}\selectfont } % Does some magic
  \renewcommand*{\printchapternum}{}
  %\hfill 
  %\fontfamily{pbk}\fontseries{m} \fontshape{n}\fontsize{80}{35}\selectfont % Makes a gigantic number
  %\chapnumfont\thechapter} % Removes the 'Chapter's number
  \renewcommand*{\printchaptertitle}[1]{%
  \noindent%
  \ifchapternonum%
  \begin{tabularx}{\textwidth}{X}%
  {\parbox[b]{\linewidth}{\raggedright \hskip -0.6em \chaptitlefont ##1}%
    \vphantom{\raisebox{0pt}{\chapnumfont 1}}}
  \end{tabularx}%
  \else

  \begin{tabularx}{\textwidth}{Xl}
  {\parbox[b]{\linewidth}{\raggedright \hskip -0.6em \chaptitlefont ##1} }
  & \raisebox{0pt}{\raggedleft \chapnumfont \hskip -0.5cm \thechapter \hskip -0.5cm}%
  \end{tabularx}%
  \fi
  \hrule
  }

  %\flushleft\chaptitlefont##1} % Prints the chaptertitle
  %\renewcommand{\afterchaptertitle}{\par\nobreak\medskip\hrule\vskip\afterchapskip}
  \setlength\beforechapskip{-100pt} % Makes the chapter go up on the page

}

\newenvironment{ffk}[0]% formel forklaring
{\begin{list}{}%
         {\setlength{\leftmargin}{\mathindent}}%
         \item[]%
}
{\end{list}}

% Akronyms formatering
\renewcommand*{\acsfont}[1]{#1}
\renewcommand*{\acffont}[1]{#1}
\renewcommand*{\acfsfont}[1]{#1}

% Farve definitioner
\definecolor{shadecolor}{gray}{.95}

\lstloadlanguages{C,VHDL,Java}
% Kodeformatering [C]
\lstnewenvironment{ccode}[2][]{
  \def\lstlistingname{Source code}
  \lstset{
    language=C,
    escapeinside={(*@}{@*)},  % a line can set with: (*@\label{c:labelname}@*)
    keywordstyle=\bfseries,
    commentstyle=\color{blue}, 
    basicstyle=\ttfamily\selectfont\footnotesize,
    numbers=left,
    numberstyle=\tiny,
    tabsize=2,
    showstringspaces=false,
    backgroundcolor=\color{shadecolor},
    frame=lines,
    captionpos=b,
    caption={#1},
    label={#2}
  }
}{}

% Kodeformatering [VHDL]
\lstnewenvironment{VHDL}[2][]{
  \def\lstlistingname{Source code}
  \lstset{
    language=VHDL,
    keywordstyle=\bfseries,
    commentstyle=\color{blue}, 
    basicstyle=\ttfamily\selectfont\small,
    numbers=left,
    numberstyle=\tiny,
    tabsize=2,
    showstringspaces=false,
    backgroundcolor=\color{shadecolor},
    frame=lines,
    captionpos=b,
    caption={#1},
    label={#2}
  }
}{}

% Kodeformatering [Assembler]
\lstnewenvironment{asmcode}[2][]{
  \def\lstlistingname{Source code}
  \lstset{
    language=[x86masm]Assembler,
    keywordstyle=\bfseries,
    commentstyle=\color{blue}, 
    basicstyle=\ttfamily\selectfont\small,
    numbers=left,
    numberstyle=\tiny,
    tabsize=2,
    showstringspaces=false,
    breaklines=true,
    backgroundcolor=\color{shadecolor},
    frame=lines,
    captionpos=b,
    caption={#1},
    label={#2}
  }
}{}

% Kodeformatering [Java]
\lstnewenvironment{javacode}[2][]{
  \def\lstlistingname{Source code}
  \lstset{
    language=Java,
    keywordstyle=\bfseries,
    commentstyle=\color{blue}, 
    basicstyle=\ttfamily\selectfont\small,
    numbers=left,
    numberstyle=\tiny,
    tabsize=2,
    showstringspaces=false,
    breaklines=true,
    backgroundcolor=\color{shadecolor},
    frame=lines,
    captionpos=b,
    caption={#1},
    label={#2}
  }
}{}

% Kodeformatering [XML]
\lstnewenvironment{xmlcode}[2][]{
  \def\lstlistingname{Source code}
  \lstset{
    language=XML,
    keywordstyle=\bfseries,
    commentstyle=\color{blue},
    basicstyle=\ttfamily\selectfont\small,
    numbers=left,
    numberstyle=\tiny,
    showstringspaces=false,
    backgroundcolor=\color{shadecolor},
    frame=lines,
    morekeywords={msgid, repeat, mmsi, navstat, rot, sog, posaccu, lon, lat, cog, truehead, timestamp, manoeuvre, raim, commstate, utc_year, utc_month, utc_day, utc_hour, utc_min, utc_sec, txbcast, name, aisver, imo, call, cargo, a, b, c, d, fixdev, eta_mon, eta_day, eta,hour, eta_min, draught, dest, dte, spare},
    captionpos=b,
    breaklines=true,
    caption={#1},
    label={#2}
  }
}{}

% \part omdefinering, nu med beskrivende tekst
\def\descpart#1#2{
  \par\newpage\clearpage % Page break 
  \vspace*{5cm} % Vertical shift 
  \refstepcounter{part}% Next part
  %\addcontentsline{toc}{part}{\texorpdfstring{\rlap{\thepart}\hspace{1.2em}#1}{\thepart\ #1}} % Adds entry to TOC 
  \addcontentsline{toc}{part}{\texorpdfstring{\rlap{\thepart}\hspace{2em}#1}{\thepart\ #1}} % Adds entry to TOC 
  {\centering \textbf{\Huge Part \thepart}\par}
  \vspace{1cm} % Vertical shift
  \thispagestyle{chapter} % Gives the pagestile an the memoir chapterpagestyle
  {\centering \textbf{\Huge #1}\par}
  \begin{center}
  {\parbox{9cm}{
  \vspace{2cm} % Vertical shift
  \noindent \slshape{#2} % Some text
  }}
  \end{center}
  \vfill\pagebreak % Fill the end of page and page break
}

%------------
\def\appendixpart#1#2{
  \par\newpage\clearpage % Page break 
  \vspace*{5cm} % Vertical shift 
  \addcontentsline{toc}{part}{\texorpdfstring{\rlap{\thepart}\hspace{2em}#1}{\thepart\ #1}} % Adds entry to TOC 
  \thispagestyle{chapter} % Gives the pagestile an the memoir chapterpagestyle
  {\centering \textbf{\Huge #1}\par}
  \begin{center}
  {\parbox{9cm}{
  \vspace{2cm} % Vertical shift
  \noindent \slshape{#2} % Some text
  }}
  \end{center}
  \vfill\pagebreak % Fill the end of page and page break
}


%------------

% At bruge figurer der fylder mere end brødteksten
% Der defineres nogle ekstra længder
\newlength{\fullwidth}
\setlength{\fullwidth}{\textwidth}
\addtolength{\fullwidth}{\marginparsep}
\addtolength{\fullwidth}{\marginparwidth}
\newlength{\fullmargin}
\setlength{\fullmargin}{\marginparwidth}
\addtolength{\fullmargin}{\marginparsep}
% For at bruge det, skal man f.eks. gøre således:
%\begin{figure}[h]
%\begin{adjustwidth*}{-\fullmargin}{}
%\includegraphics[width=\fullwidth]{./sti}
%\end{adjustwidth*}
%\caption{Beskrivelse her}
%\laber{fig:label}
%\end{figure}

% \degC for grader C
% \arcdeg for gradtegn
\renewcommand{\epsilon}{\varepsilon}
\newcommand{\MATLAB}{M{\footnotesize ATLAB}}
\parindent 8pt

% Orddeling
% Ord der deles forkert kan skrives her, men bindestreg ved stavelser og mellemrum mellem ordene
\hyphenation{ar-bejds-in-ten-si-ve an-grebs-vink-ler ind-gangs-im-pe-dans ro-ta-ti-ons-en-ko-der deres clock-fre-kvens sys-tem-et na-vi-gate} 
