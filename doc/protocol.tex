\documentclass{article}
% Pakker
\usepackage[utf8]{inputenc} % Så må vi bruge æ, ø og å
%\usepackage[ansinew]{inputenc}
%\usepackage[danish]{babel} % Dansk opsætning
\usepackage[T1]{fontenc} % Hjælper med ordeling ved æ, ø og å. Sætter fontene til at være ps-fonte i stedet for bmp.
\usepackage{array,booktabs} % Til gode tabeller
\usepackage{acronym} % Smart akronymhåndtering
\usepackage{minitoc} % Vi kan lave del inholdsfortegnelser forhåbentlig
\usepackage{bytefield}
\begin{document}
\section{FAPS LLI interface}
\section{Standard format of messages}
\begin{bytefield}{15}
\bitbox{5}{DevID} & \bitbox{5}{MsgID} 
\bitbox{16}{Data } \quad \raisebox{2ex}{\dots}  \bitbox{4}{CRC16 }
\end{bytefield}

Little endian...

\begin{table}
	\centering
	\begin{tabular}{lll}
		\toprule
		\textbf{Field name} & \textbf{Size [bytes]} & \textbf{Description}\\
		\midrule
		startchar & 1 (uint8) & Start character (\texttt{\$}) \\
		devid & 1 (uint8) & Device identifier \\
		msgid & 1 (uint8) & Message identifier \\
		data & 1--250 (uint8) & Data portion (binary)\\
		checksum & 2 (uint8) & CRC-16 checksum on data part \\
		endchar & 1 (uint8) & Newline character (\texttt{\textbackslash n})\\
		\bottomrule
	\end{tabular}
	\caption{General description of the packet format}
\end{table}
\end{document}
